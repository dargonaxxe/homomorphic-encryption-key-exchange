% This file demonstrates how to use the IEEEConf LaTeX2e macro package,
% to prepare a manuscript for proceedings on CD of the conference
% FedCSIS
%
\documentclass[conference]{IEEEtran}
%\documentclass[a4paper]{IEEEconf}

% This package serves to balance the column lengths on the last page of the document.
% please, insert \balance command in the left column of the last page
\usepackage{balance}
\usepackage{amsfonts}
\usepackage{hyperref}
\usepackage{amsthm}
\theoremstyle{definition}
\newtheorem{definition}{Definition}[section]
\providecommand{\keywords}[1]{\textbf{\textit{Keywords---}} #1}

%% to enable \thank command
\IEEEoverridecommandlockouts 
%% The usage of the following packages is recommended
%% to insert graphics
\usepackage[dvips]{graphicx}
% to typeset algorithms
\usepackage{algorithmic}
\usepackage{cite}
\usepackage{algorithm}
% to typeset code fragments
\usepackage{listings}
% to make an accent \k be available
\usepackage[OT4,T1]{fontenc}
% provides various features to facilitate writing math formulas and to improve the typographical quality of their output.
\usepackage[cmex10]{amsmath}
\usepackage{amsthm}
\interdisplaylinepenalty=2500
% por urls typesetting and breaking
\usepackage{url}
% for vertical merging table cells
\usepackage{multirow}

% define environments for remarks and examples
\newtheorem{remark}{Remark}[section]
\newtheorem{example}[remark]{Example}


\title{Key Exchange Algorithm Based on Homomorphic Encryption}

\author{
	\IEEEauthorblockN{Sergei Krendelev}
	\IEEEauthorblockA{
		Novosibirsk State University \\
		JetBrains Research Cryptographic Lab\\
		s.f.krendelev@gmail.com
	}
	
	\and	
	\IEEEauthorblockN{Ilya Kuzmin}
	\IEEEauthorblockA{
		Novosibirsk State University \\
		JetBrains Research Cryptographic Lab \\
		dargonaxxe@gmail.com	
	}

%	\and 
%	\IEEEauthorblockN{Nikita Sarin}
%	\IEEEauthorblockA{
%		Novosibirsk State University \\
%		idnikitasarin@gmail.com
%	}
}

% for over three affiliations, or if they all won't fit within the width
% of the page, use this alternative format:
% 
%\author{\IEEEauthorblockN{Michael Shell\IEEEauthorrefmark{1},
%Homer Simpson\IEEEauthorrefmark{2},
%James Kirk\IEEEauthorrefmark{3}, 
%Montgomery Scott\IEEEauthorrefmark{3} and
%Eldon Tyrell\IEEEauthorrefmark{4}}
%\IEEEauthorblockA{\IEEEauthorrefmark{1}School of Electrical and Computer Engineering\\
%Georgia Institute of Technology,
%Atlanta, Georgia 30332--0250\\ Email: see http://www.michaelshell.org/contact.html}
%\IEEEauthorblockA{\IEEEauthorrefmark{2}Twentieth Century Fox, Springfield, USA\\
%Email: homer@thesimpsons.com}
%\IEEEauthorblockA{\IEEEauthorrefmark{3}Starfleet Academy, San Francisco, California 96678-2391\\
%Telephone: (800) 555--1212, Fax: (888) 555--1212}
%\IEEEauthorblockA{\IEEEauthorrefmark{4}Tyrell Inc., 123 Replicant Street, Los Angeles, California 90210--4321}}





\begin{document}
	\maketitle              % typeset the title of the contribution
	
	\begin{abstract}
	Key exchange algorithm based on homomorphic encryption ideas is reviewed in this article. This algorithm might be used for safe messaging using one-time pads. Since algorithm requires the low amount of computing resources, this method might be used in IoT to provide the authentication.
\end{abstract}
	
	\keywords{\textbf{\textit{homomorphic encryption; key exchange; one time pad} } }	
	
	\section{Introduction}
	Expected quantum computer evolution causes the intensive development of cryptographic primitives called postquantum cryptography. February 6, 2016, the day \textit{National Institute of Standards and Technology} (NIST) offered to start the development of new postquantum cryptography standards which might be used in governmental needs. Accordingly to documents submitted by NIST, algorithms based on discrete logarithm problem are vulnerable to quantum attacks. Moreover, elliptical cryptography methods are considered to be vulnerable. Therefore, we need to replace the Diffie-Hellman key exchange algorithm in TLS protocol. \\ 
Nowadays offered postquantum key exchange algorithms are based on lattice theory (LWE, RLWE), which is used in key exchange algorithms named New hope and Frodo. Those algorithms are being supported by Google as TLS postquantum update. \\
We represent the key exchange algorithm based on basic homomorphic encryption properties and linear algebra methods. The main purpose of this algorithm is to ensure the secure messaging. It's assumed that one-time pad method will be used. For being used in TLS the algorithm should be changed in the proper way. 
	
	\section{Notation}
	In this section we will describe the notation that will be used futher. \\
	Let $ \mathbb{Z} $ be a ring of integer numbers. For $ n \in \mathbb{N} $ let's call $ \textbf{a} = (a_1, \dots, a_n) $ \textit{$ n $-dimensional integer vector} if $ \forall i \in \{1, \dots, n\} a_i \in \mathbb{Z} $. For $ n \in \mathbb{N} $ let $ \mathbb{Z}^n $ denote the set of all possible $ n $-dimensional integer vectors. Moreover, it's assumed that $ n $-dimensional vectors have the following properties:
	\begin{enumerate}
		\item For each pair $ \textbf{x}, \textbf{y} \in \mathbb{Z}^n $  the following is true $ \textbf{x} + \textbf{y} = (x_1 + y_1, \dots, x_n + y_n) $.
		\item For each $ \textbf{x} \in \mathbb{Z}^n, \alpha \in \mathbb{Z} $ the following is true $ \alpha \textbf{x} = (\alpha x_1, \dots, \alpha x_n) $.
	\end{enumerate}
	Moreover, for $ \textbf{x}, \textbf{y} \in \mathbb{Z}^n $ let's call $ \textbf{x} \cdot \textbf{y} = \sum\limits_{i = 1}^n x_i y_i $ a \textit{dot product}. \\
	Let $ \chi $ be a probability distribution over $ R $. Accordingly, $ x \leftarrow \chi $ denotes sampling $ x \in R $ according to $ \chi $. Moreover, for a finite set $ A $, let $ U(A) $ denote the uniform distribution on $ A $ and $ x \leftarrow U(A) $ sampling $ x $ unifromly at random from $ A $.
	\begin{definition}{Homomorphic ecnryption}
		Let $ \textbf{x} \in \mathbb{Z}^n $ be a fixed vector for some $ n \in \mathbb{N} $. Moreover, let's consider that $ \textbf{x} $ has \textit{at least} 2 coprime components. \\
		For number $ d \in \mathbb{Z} $ and vector $ \textbf{x} $ we will call a vector $ \textbf{a} \in \mathbb{Z}^n $ an \textit{interpretation} if $ \textbf{x} \cdot \textbf{a} = d $. \\
		Therefore we have a mapping $ \Phi_{\textbf{x}}: \mathbb{Z} \rightarrow \mathbb{Z}^n $. Easy to notice that this mapping has the following properties:
		\begin{enumerate}
			\item $ \Phi_{\textbf{x}}(\textbf{a}_1 + \textbf{a}_2) = \Phi_{\textbf{x}}(\textbf{a}_1) + \Phi_{\textbf{x}}(\textbf{a}_2) $
			\item $ \Phi_{\textbf{x}}(\alpha \textbf{a}) = \alpha \Phi_{\textbf{x}}(\textbf{a}) $
		\end{enumerate}
		This mapping is called \textit{homomorphic encryption}.
	\end{definition}

			
	
	\section{Basic homomorphic encryption based key exchange}
	In this section we review the basic key exchange based on homomorphic encryption and detect it's vulnerability. \\
	Suggested algorithm is an analogue of 
	Let's suppose, two users -- Alice and Bob have vectors $ \textbf{x}, \textbf{y} \in \mathbb{Z}^k $
	
	\section{MITM passive attack}
In this section we estimate how successful a \textit{Man In The Middle} (MITM) passive attack can be. Passive means that an adversary can't edit the data transmitted by Alice and Bob. \\
An adversary has vectors $ \textbf{a}_1, \dots, \textbf{a}_k, \textbf{b}_1, \dots, \textbf{b}_k $, $ \textbf{s}_1, \dots, \textbf{s}_p, \textbf{r}_1, \dots, \textbf{r}_q $. Also, the following system of equations is known by adversary:
\begin{center}
	$\textbf{w} = (\textbf{b}_1 \cdot \textbf{y}) \textbf{a}_1 + \dots + (\textbf{b}_k \cdot \textbf{y}) \textbf{a}_k + \lambda_1 \textbf{s}_1 + \dots + \lambda_p \textbf{s}_p$ \\
	$ \textbf{v} = (\textbf{a}_1 \cdot \textbf{x}) \textbf{b}_1 + \dots + (\textbf{a}_k \cdot \textbf{x}) \textbf{b}_k + \mu_1 \textbf{r}_1 + \dots + \mu_q \textbf{r}_q $ \\
	$\textbf{s}_1 \cdot \textbf{x} = 0, \dots, \textbf{s}_p \cdot \textbf{x} = 0$ \\
	$\textbf{r}_1 \cdot \textbf{y} = 0, \dots, \textbf{r}_q \cdot \textbf{y} = 0 $
\end{center}
Choosing proper values for $ p, q, k, m, n $, users can make the system underdetermined. Hence the needed solution can't be found by adversary. To improve the algorithm users can substantially increase dimension using sparse vectors. 
	
	\section{Toy example}
In this section we reduce the number of dimensions to show the way algorithm works. \\
Let $ k = 4, n = 3, m = 2, p = 2, q = 2, z = 0, z' = 7 $. \\
\begin{enumerate}
	\setcounter{enumi}{-1}
	\item Alice and Bob chooses $ z = 0, z' = 7, k = 4 $.
	\item Alice chooses $ n = 3 $, $ \textbf{x} = \begin{pmatrix} 2 & 3 & 4 \end{pmatrix}	 $,\\
	 $ \textbf{a}_1 = \begin{pmatrix} 4 & 3 & 7 \end{pmatrix} $, \\
	 $ \textbf{a}_2 = \begin{pmatrix} 3 & 0 & 1 \end{pmatrix} $, \\ 
	 $ \textbf{a}_3 = \begin{pmatrix} 3 & 5 & 3 \end{pmatrix} $, \\
	 $ \textbf{a}_4 = \begin{pmatrix} 1 & 3 & 7 \end{pmatrix} $. \\
	 $ d_1 = 45, d_2 = 10, d_3 = 33, d_4 = 39 $. \\
	 $ \textbf{s}_1 = \begin{pmatrix} -3 & 2 & 0 \end{pmatrix}$\\
	 $ \textbf{s}_2 = \begin{pmatrix} 0 & -4 & 3 \end{pmatrix} $. 
	\item Bob chooses $ m = 2 $, $ \textbf{y} = \begin{pmatrix} 1 & 5 \end{pmatrix} $,\\
	$ \textbf{b}_1 = \begin{pmatrix} 6 & 5 \end{pmatrix}$,\\
	$\textbf{b}_2 = \begin{pmatrix} 6 & 6 \end{pmatrix}$,\\ 
	$\textbf{b}_3 = \begin{pmatrix} 5 & 7 \end{pmatrix}$,\\
	$\textbf{b}_4 = \begin{pmatrix} 5 & 4 \end{pmatrix} $. \\
	$ h_1 = 31, h_2 = 36, h_3 = 40, h_4 = 25 $. \\
	$ \textbf{r}_1 = \begin{pmatrix} -5 & 1 \end{pmatrix}$, \\ 
	$ \textbf{r}_2 = \begin{pmatrix} 10 & -2 \end{pmatrix} $.
	\item Alice chooses $ \mu_1 = 6, \mu_2 = 5 $, then calculates 
	$ \textbf{v} = 45 \textbf{b}_1 + 10 \textbf{b}_2 + 33 \textbf{b}_3 + 39 \textbf{b}_4 + 6 \textbf{r}_1 + 5 \textbf{r}_2 = \begin{pmatrix} 710 & 668 \end{pmatrix}	 $.
	\item Bob chooses $ \lambda_1 = 7, \lambda_2 = 3 $, then calculates \\
	$ \textbf{w} = 31 \textbf{a}_1 + 36 \textbf{a}_2 + 40 \textbf{a}_3 + 25 \textbf{a}_4 + 7 \textbf{s}_1 + 3 \textbf{s}_2 = \begin{pmatrix} 356 & 370 & 557 \end{pmatrix} $.
	\item Alice calculates $ \textbf{x} \cdot \textbf{w} = 4050 $ 
	\item Bob calculates $ \textbf{y} \cdot \textbf{v} = 4050 $
\end{enumerate} 
	
	\section{Implementation and performance}
	We implemented this algorithm using the C++ programming laguage. Implementation\footnote{\href{https://github.com/dargonaxxe/homomorphic-encryption-key-exchange}{github.com}} uses open source long arithmetics library GNU MP (GMP). The values for $ k, n, m, p, q $ are fixed: $ k = 60, n = 45, m = 40, p = 30, q = 35 $. Table 1 represents the average results of 400 tests being executed on the single PC with CPU Intel Core i7-640M Processor with 4M Cache, 2.80 GHz. Here what represents each row:
	\begin{enumerate}
		\item Key size -- amount of bits required to contain the key in memory. 
		\item Time spent on initialization -- time taken by Alice to perform part 1 of algorithm (part 2 for Bob respectively).
		\item Time spent on calculation -- time taken by Alice to perform calculations from part 3 and 5 (4 and 6 for Bob respectively).
		\item Time spent on data transmission -- this time is a theoretical value. We calculated it assuming that both users have stable Internet connection of 50 Mbps.
		\item Time spent on key exchange with preparation -- time taken by user to perform the key exchange assuming that user already generated the data from part 1-2. 
		\item Time spent on key exchange without preparation -- the opposite, time taken by user to perform key exchange with data generation.
		\item Amount of transmitted data -- size of messages sent by users. 
	\end{enumerate}
	
\begin{center}

\begin{table}
\caption{Performance} \label{table}
\centering
\begin{tabular}{|c|c|c|}
	\hline
	\hfill & Alice & Bob \\
	\hline
	Key size & \multicolumn{2}{|c|}{2290 bits} \\
	\hline
	
	\begin{tabular}{c}
	Time spent on initialization
	\end{tabular} 						& 130 ms & 108.5 ms \\
	\hline
	
	\begin{tabular}{c}
	Time spent on calculation
	\end{tabular}						& 5.2 ms & 5.6 ms \\
	\hline
	
	\begin{tabular}{c}
	Time spent on data transmission
	\end{tabular}						& 2.9 ms & 2.9 ms \\
	\hline
	
	\begin{tabular}{c}
	Time spent on key exchange \\ with preparation
	\end{tabular}						& 8.1 ms & 8.5 ms \\
	\hline
	
	\begin{tabular}{c}
		Time spent on key exchange \\ without preparation	
	\end{tabular}						& 138.2 ms & 117.1 ms \\
	\hline
	
	Amount of transmitted data & 19208 bytes & 19016 bytes \\
	\hline

\end{tabular}
\end{table}

\end{center}
	
	\section{Conclusion}
	The reviewed algorithm is a promising cryptographic primitive that is believed to be resistant to quantum attacks. The implementation results are presented in Table 1. The benchmarking results are measured on Intel Core i7-640M with 2 cores running at 2.8 GHz. The implementation details are shown in GitHub repository\footnote{\href{https://github.com/dargonaxxe/homomorphic-encryption-key-exchange}{github.com/dargonaxxe/homomorphic-encryption-key-exchange}}
	

	\bibliography{fedcsis.bib}{}
	\bibliographystyle{plain}

\end{document}
