\section{Introduction}
	\IEEEoverridecommandlockouts\IEEEPARstart{T}{he expected} evolution of quantum computers is causing the intensive development of cryptographic primitives called postquantum cryptography. February 6, 2016 was the day that the \textit{National Institute of Standards and Technology} (NIST) offered to start the development of new postquantum cryptography standards which might be used in governmental needs. According to documents submitted by NIST, algorithms based on a discrete logarithmic problem are vulnerable to quantum attacks. Moreover, elliptical cryptography methods are considered to be vulnerable. Therefore, we need to replace the Diffie-Hellman key exchange algorithm in TLS protocol. \\ 
Nowadays, offered postquantum key exchange algorithms are based on lattice theory \cite{Peikert2014} (LWE, RLWE \cite{7163047}), which is used in key exchange algorithms named New hope\cite{cryptoeprint:2015:1092} and Frodo\cite{cryptoeprint:2016:659}. These algorithms are being supported by Google as TLS postquantum update. \\
We represent the key exchange algorithm based on basic homomorphic encryption properties and linear algebraic methods. The main purpose of this algorithm is to ensure secure messaging. It's assumed that the one-time pad method will be used. To use the algorithm in TLS one should change it in accordance with specification.