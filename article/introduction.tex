\section{Introduction}
	Expected quantum computer evolution causes the intensive development of cryptographic primitives called postquantum cryptography. February 6, 2016, the day \textit{National Institute of Standards and Technology} (NIST) offered to start the development of new postquantum cryptography standards which might be used in governmental needs. Accordingly to documents submitted by NIST, algorithms based on discrete logarithm problem are vulnerable to quantum attacks. Moreover, elliptical cryptography methods are considered to be vulnerable. Therefore, we need to replace the Diffie-Hellman key exchange algorithm in TLS protocol. \\ 
Nowadays offered postquantum key exchange algorithms are based on lattice theory (LWE, RLWE), which is used in key exchange algorithms named New hope and Frodo. Those algorithms are being supported by Google as TLS postquantum update. \\
We represent the key exchange algorithm based on basic homomorphic encryption properties and linear algebra methods. The main purpose of this algorithm is to ensure the secure messaging. It's assumed that one-time pad method will be used. For being used in TLS the algorithm should be changed in the proper way. 