\section{Notation}
	In this section we will describe the notation that will be used futher. \\
	Let $ \mathbb{Z} $ be a ring of integer numbers. Let $ \mathbb{Z}_q = \mathbb{Z} / q\mathbb{Z} $ be the quotient ring of any integer $ q \geq 1 $. Let $ q $ be some fixed integer number such that $ q \geq 1 $ i.e. $ q = 2^{64} $. Let $ R = \mathbb{Z}_q $. Accordingly, if $ k \in \mathbb{N} $, let $ R^k $ denote the set of vectors with $ k $ components from $ R $ i.e.  $ R^1 = R $. For $ \textbf{a}, \textbf{b} \in R^k, \textbf{a} = (a_1, \dots, a_k), \textbf{b} = ( b_1, \dots, b_k ) $ let's denote $ \textbf{a} \cdot \textbf{b} = \sum_{i = 1}^{k} a_i b_i $ as a \textit{dot product}. \\
	Let $ \chi $ be a probability distribution over $ R $. Accordingly, $ x \leftarrow \chi $ denotes sampling $ x \in R $ according to $ \chi $. Moreover, for a finite set $ A $, let $ U(A) $ denote the uniform distribution on $ A $ and $ x \leftarrow U(A) $ sampling $ x $ unifromly at random from $ A $.
	\begin{definition}{Homomorphic ecnryption}
		Let $ \textbf{x} \in \mathbb{Z}^n $ be a fixed vector for some $ n \in \mathbb{N} $. Moreover, let's consider that $ \textbf{x} $ has \textit{at least} 2 coprime components. \\
		For number $ d \in \mathbb{Z} $ and vector $ \textbf{x} $ we will call a vector $ \textbf{a} \in \mathbb{Z}^n $ an \textit{interpretation} if $ \textbf{x} \cdot \textbf{a} = d $. \\
		Therefore we have a mapping $ \Phi_{\textbf{x}}: \mathbb{Z} \rightarrow \mathbb{Z}^n $. Easy to notice that this mapping has the following properties:
		\begin{enumerate}
			\item $ \Phi_{\textbf{x}}(\textbf{a}_1 + \textbf{a}_2) = \Phi_{\textbf{x}}(\textbf{a}_1) + \Phi_{\textbf{x}}(\textbf{a}_2) $
			\item $ \Phi_{\textbf{x}}(\alpha \textbf{a}) = \alpha \Phi_{\textbf{x}}(\textbf{a}) $
		\end{enumerate}
		This mapping is called \textit{homomorphic encryption}.
	\end{definition}
		