\section{Notation}
	In this section we will describe the notation that will be used futher. \\
	Let $ \mathbb{Z} $ be a ring of integer numbers. For $ n \in \mathbb{N} $ let's call $ \textbf{a} = (a_1, \dots, a_n) $ \textit{$ n $-dimensional integer vector} if $ \forall i \in \{1, \dots, n\} a_i \in \mathbb{Z} $. For $ n \in \mathbb{N} $ let $ \mathbb{Z}^n $ denote the set of all possible $ n $-dimensional integer vectors. Moreover, it's assumed that $ n $-dimensional vectors have the following properties:
	\begin{enumerate}
		\item For each pair $ \textbf{x}, \textbf{y} \in \mathbb{Z}^n $  the following is true $ \textbf{x} + \textbf{y} = (x_1 + y_1, \dots, x_n + y_n) $.
		\item For each $ \textbf{x} \in \mathbb{Z}^n, \alpha \in \mathbb{Z} $ the following is true $ \alpha \textbf{x} = (\alpha x_1, \dots, \alpha x_n) $.
	\end{enumerate}
	Moreover, for $ \textbf{x}, \textbf{y} \in \mathbb{Z}^n $ let's call $ \textbf{x} \cdot \textbf{y} = \sum\limits_{i = 1}^n x_i y_i $ a \textit{dot product}. \\
	Let $ \chi $ be a probability distribution over $ \mathbb{Z} $. Accordingly, $ x \leftarrow \chi $ denotes sampling $ x \in \mathbb{Z} $ according to $ \chi $. Moreover, for $ a, b \in \mathbb{Z}, a < b $ let $ x \leftarrow U_{a, b} $ denote sampling $ x $ uniformly from $ \{a, a+1, \dots, b\} $. Moreover, let $ \textbf{x} \leftarrow U_{a, b}^n $ denote sampling $ \textbf{x} $ in the following way: $ \textbf{x} = (x_1 \leftarrow U_{a, b}, \dots, x_n \leftarrow U_{a, b} ). $
	\begin{definition}{Homomorphic ecnryption}
		Let $ \textbf{x} \in \mathbb{Z}^n $ be a fixed $ n $-dimensional integer vector for some $ n \in \mathbb{N} $. Moreover, let's consider that $ \textbf{x} $ has \textit{at least} 2 coprime components. \\
		For number $ d \in \mathbb{Z} $ and vector $ \textbf{x} $ we will call a vector $ \textbf{a} \in \mathbb{Z}^n $ an \textit{interpretation} if $ \textbf{x} \cdot \textbf{a} = d $. \\
		Therefore we have a mapping $ \Phi_{\textbf{x}}: \mathbb{Z} \rightarrow \mathbb{Z}^n $. Easy to notice that this mapping has the following properties:
		\begin{enumerate}
			\item $ \Phi_{\textbf{x}}(\textbf{a}_1 + \textbf{a}_2) = \Phi_{\textbf{x}}(\textbf{a}_1) + \Phi_{\textbf{x}}(\textbf{a}_2) $
			\item $ \Phi_{\textbf{x}}(\alpha \textbf{a}) = \alpha \Phi_{\textbf{x}}(\textbf{a}) $
		\end{enumerate}
		This mapping is called \textit{homomorphic encryption}.
	\end{definition}

		